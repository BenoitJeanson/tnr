\section{Conclusion}
Noting that the N-1 rule is fundamentally a risk-based approach, our approach highlights situations that necessary happen in transmission grid, and particularly in subtransmission systems where the risk of loosing parts of the grid after a single contingency is considered. This operational policy is modeled as an MILP, which leverages connectivity analysis in the N-1 cases that is incorporated into the OTS problem. As a result, solutions similar to those observed in operation are reached by the algorithm, notably revealing the preventive-openings-cascade phenomenon. The application of such an approach would benefit the operator, as the design of these strategies is time consuming. 

In future work, this approach will be extended with bus splitting actions that are generally preferred to line openings, AC power flow to take into account reactive power and voltage limitations, and completed with an Optimal Power Flow to include the redispatching lever. Finally, the algorithm must scale for solving the problem on real-size grids.