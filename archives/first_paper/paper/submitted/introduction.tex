\section{Introduction}

One of the primary concerns of Transmission System Operators (TSOs) is ensuring the security of the power system. To achieve this, they must anticipate potential events that can compromise it while simultaneously striving to maintain economic efficiency.
When it comes to security policies, the cornerstone is the N-1 rule. Sometimes, it is
reduced to the statement that the consequences of the tripping of any single element
of the grid shall have no impact on the users of the power system. This is
actually a narrow view. In fact, the rule derives from a risk-based principle. As
contingencies are likely to occur, what is at stake for the operator is that any resulting
states of the system remain under control. In other words, the consequences of
any likely tripping of a branch shall be mastered. Furthermore, to align the risks to the
same level, the more likely a contingency is, the less acceptable its
consequences in case of occurrence. So, the criteria the operator follows
are based on the definition of the limits between an acceptable and an unacceptable
consequence with respect to each contingency considered.

Keeping flows or voltages within security limits is one of the key
criteria. Losing a part of the grid and de-energizing users could also be
considered unacceptable, but is it realistic? In fact, enforcing the latter constraint
implies that the real-time operator must avoid \emph{at all costs} losing customers
as a result of a single tripping. This may involve operations on the grid itself,
for example, by implementing switching actions or activating redispatch or curtailment of customers that rapidly becomes expensive. Since the probability of tripping
is uncertain but usually small, the associated costs can be viewed as an investment in insurance
to mitigate the potential consequences.

It should also be noted that the EU regulation \cite{eu-2017/1485} does not contain such
standards. \emph{Loss} is only explicitly mentioned in the description of the security
states, which are: normal, alert, emergency, blackout. It is introduced
for the blackout state, which is defined by the \emph{loss} of more than 50\% of
the total consumption of the control area of the TSO. The fact that there is no specific
standard reflects at least that there is no requirement to not lose any part of the grid,
and opens to the interpretation that the regulation admits this may happen. Moreover, there may be operational scenarios where solutions that avoid all customer disconnections do not exist. 


This paper considers situations where no solution is available, or where the costs associated with the mitigation of potential loss of parts of the grid are prohibitive. Such cases
are addressed in the context of the Optimal Transmission Switching (OTS) problem.

The OTS problem consists in finding the combination of branch openings in the grid
that optimizes a given objective function. Many optimization approaches to the
OTS problem have been developed. \cite{fisherOptimalTransmissionSwitching2008} by
Fisher~\emph{et al.}~is the first DC-OPF (Direct Current Optimal Power Flow) based on an MILP (Mixed-Integer
Linear Program) approach to it with the goal of solving \emph{congestion}.
This pioneering study lays the foundation for numerous subsequent efforts that employ
the DC approximation to solve the OTS. In this model, the switching problem is
formulated using a big-M approach and the necessary flexibility -- when no
switching scheme is enough to cope with the congestion -- is provided by the OPF
equations that aim at minimizing generation adjustment cost. Hedman~\emph{et al.}~extended
that approach in \cite{hedmanOptimalTransmissionSwitching2009} to include \emph{security
analysis} and incorporate the N-1 rule. The algorithm does not permit the loss of load or generation, but it does allow electrical islands either in the
base case or after a contingency if the resulting islands satisfy all the constraints.
 In practical operation, islanding should
only be considered in very particular power systems that are designed to be sustainable
after a grid split, which implies the fulfillment of a lot of
requirements in terms of balancing, control, and stability. To avoid islanding, various recent papers therefore  take into account a connectivity constraint, for which various approaches were proposed (\cite{ostrowskiTransmissionSwitchingConnectivityEnsuring2014}, \cite{hanEnsuringNetworkConnectedness2021}, \cite{liConnectivityConstrainedMILP2021}). 

The contribution of this article is threefold. First, a discussion of the N-1 rule
is presented and consequences of including it in operation are presented. Then
a risk-based optimization model that reflects that operational policy is developed. At its
core, the connectedness of the network in the base case and in the N-1
situations is assessed. Finally, its application to the IEEE 14-bus system is
analyzed.