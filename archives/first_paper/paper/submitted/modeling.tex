\section{MILP modeling}
\subsection{Principles}
This section develops an MILP formulation of the approach described above. 
The aim is to determine preventive branch openings, if any, that will optimally reduce system risk. These are described by a binary vector $\pmb{v}$, indicating the pre-contingency status of each branch (1 is open). 

The base case is supposed to be balanced and the model shall ensure its connectedness
(\ref{ss:base case connectedness}). On the other hand, in N-1 the connectedness is
assessed to identify for each contingency which part of the grid, if any, is deenergized
(\ref{ss:N-1 Connectedness}). The generators remaining in the energized area are
adjusted to ensure the balance of the N-1 case (\ref{ss:balancing}).
Finally, a secured DC power flow (\ref{ss:dcpf}) is modeled to assess the flows in
the branches, and constraints are added to include their flow limitations. The cost
function aiming at minimizing the risk taken is described in \ref{ss:cost function}.

Some formulations are not directly applicable as MILP for they involve logical operations
or Hadamard product of variables. They are first exposed in their concise form
for simplicity and readability, and the steps used to translate them into MILP formulations are developed in \ref{ss:bigM}.

\subsection{Base case connectedness}
\label{ss:base case connectedness} In order to ensure the connectedness of the grid
in the base case, (\ref{eq:base case connectedness}) implements a method inspired by \cite{hanEnsuringNetworkConnectedness2021} involving
a fictitious mirror graph. A subscript $0$ is used to indicate the pre-contingency base case.

\begin{subequations}
    \allowdisplaybreaks
    \begin{align}
        \pmb{f}^{\star}_{0}\odot \pmb{v} & = 0 \label{subeq:connecteness branch opening}                     &                                                                                      \\
        \pmb{A}\pmb{f}^{\star}_{0}       & = \pmb{p}^{\star}_{0}\label{subeq:base connectedness conservation} \\
        \
 p^{\star}_{0,i}               & = 1                                                               & \quad \forall i \in \mathcal{V}\backslash \{s\} \label{subeq:base connectednes load} \\
        p^{\star}_{0,s}                  & = - (\left|\mathcal{V}\right| - 1)                                & \label{subeq:base connectedness source}
    \end{align}
    \label{eq:base case connectedness}
\end{subequations}

The mirror graph has the same vertices and edges as the graph of the grid. If one branch is open in the grid, the corresponding edge is open in
the mirror graph, and its flow is null (\ref{subeq:connecteness branch opening}).
In this mirror graph, the only law that applies is the principle of conservation (\ref{subeq:base
connectedness conservation}). One vertex with the index $s$ is chosen as the
source -- a virtual generator. All others consume $1$ (\ref{subeq:base
connectednes load}). Thus, if the graph is connected, there is a connected path between
the source vertex and any of the others, and by application of the principle of
conservation, the source vertex shall produce $\left |\mathcal{V}\right|-1$,
which is ensured by (\ref{subeq:base connectedness source}). Only a connected graph can satisfy these constraints.

\subsection{N-1 Connectedness handling}
\label{ss:N-1 Connectedness} 
We consider only branch opening contingencies; hence we use the convention that contingency $c\in\mathcal{C}^{*}$ corresponds to opening of branch $c$. A new binary vector $\pmb{w_c}$ is introduced that combines
the switching status of the branches for the contingency $c$. $w_{c,e}$ has value 1 if the edge $e$ is open either preventively by the OTS or because of the contingency; it is 0 otherwise:
\begin{equation}
    w_{c,e}= v_{e}
    \vee \delta_{c,e} \quad \quad  \forall \left(c,e\right) \in \mathcal{C}\times\mathcal{V}
\end{equation}
Note that for index $c=0$ (the base case), $\pmb{w}_{0}=\pmb{v}$.

In the risk-based approach, after a contingency, part of the grid may be
deenergized. As for the base case connectedness, a fictitious mirror graph is
implemented (\ref{eq:OTS N-1 connectedness}) for each considered contingency in
$\mathcal{C}$. However, the goal is no longer to ensure connectedness, but rather
to identify the buses that would be disconnected from the Main Connected Component
(MCC) of the grid after the trip. The MCC is defined as the set of nodes connected to the mirror grid source vertex. For simplicity, the same vertex $s$ chosen for the base case is used here, but more elaborate definitions could be used, including the use of a contingency-specific source vertex. 

$\forall c \in \mathcal{C}^{*}$:
\begin{subequations}
    \allowdisplaybreaks
    \begin{align}
        \pmb{f}_{c}^{\star}\odot \pmb{w}_{c}     & = \pmb{0}\label{subeq:N-1 connecteness branch opening}                           \\
        \pmb{A}\pmb{f}_{c}^{\star}               & = \pmb{p}^{\star}_{c}\label{subeq:N-1 connectedness conservation}                \\
        \pi_{c,i}                                & \in \{0, 1\}                                                                    & \forall i \in \mathcal{V}\label{subeq:N-1 connectedness pi declaration}  \\
        \pi_{c,s}                                & = 1 \label{subeq:N-1 connectedness set source}                                   \\
        p^{\star}_{c, s}                         & \in \{-(|\mathcal{V}| - 1) , ..., 0\} \label{subeq:N-1 connectedness source set} \\
        p^{\star}_{c,i}                          & = \pi_{c,i}                                                                     & \forall i \in \mathcal{V}\backslash \{s\} \label{subeq:N-1 pi consumers} \\
        \pi_{c,i}= \bigvee_{e \in \text{inc}(i)} & \left[ \pi_{c,\text{opp}(i,e)}\wedge(1-w_{c,e})  \right]                                      & \forall i \in \mathcal{V}\label{subeq:N-1 connectedness pi disjunction}
    \end{align}
    \label{eq:OTS N-1 connectedness}
\end{subequations}

Equations (\ref{subeq:N-1 connecteness branch opening}) and (\ref{subeq:N-1
connectedness conservation}) play the same role as (\ref{subeq:connecteness
branch opening}) and (\ref{subeq:base connectedness conservation}). To identify
the part of the grid that remains in the main connected component, an indicator variable
$\pi_{c,i}$ is introduced that is set to $1$ for the contingency $c$ if the vertex
$i$ is energized (\ref{subeq:N-1 connectedness pi declaration}), $0$ otherwise. Its value
for the source vertex is set to $1$ as it is per definition in the MCC (\ref{subeq:N-1
connectedness set source}). Now, a similar approach to that of the base case is applied
where the source vertex $s$ is the only one that holds a generator (\ref{subeq:N-1
connectedness source set}), and all the other vertices a consumption set to the
value of $\pi_{c,i}$ (\ref{subeq:N-1 pi consumers}). So, when the vertex $i$ is
not connected to the MCC, as there is no connected path to the source vertex,
the vertex cannot consume and $\pi_{c,i}=0$. In fact, having $p^{\star}_{c,i}=1$
would contradict \ref{subeq:N-1 connectedness conservation} in the
de-energized area.
This only ensures that $\pi_{c,i}$ is set to 0 for de-energized buses. The next mechanism is
necessary to force it to be $1$ in the energized area. Starting from the source
vertex for which $\pi_{c,s}=1$, the status propagates by (\ref{subeq:N-1
connectedness pi disjunction}). 
Each bus $i$ must be energized if at least one of its connected ($w_{c,e}=0$) neighbors is also energized. 

\subsection{Balancing the energized area}\label{ss:balancing}
If a tripping leads to a de-energized area ($\pi_{c, i}$), then the generation and load in the remaining nodes must be
balanced to continue operation. 
We choose to balance by shifting all remaining energized generators
proportionally to their initial values:

$\quad \forall c \in \mathcal{C}^{*}$:
\begin{subequations}
    \begin{align}
        \pmb{\hat{d}}_{c}                                          & = \pmb{d}\odot \pmb{\pi}_{c}\label{subeq:balancing load}                 \\
        \pmb{\hat{g}}_{c}                                          & = \sigma_{c}\pmb{g}\odot \pmb{\pi}_{c}\label{subeq:balancing generation} \\
        \langle\pmb{1}, \hat{\pmb{g}}_{c}- \hat{\pmb{d}}_{c}\rangle & = 0 \label{subeq:balancing overall}
    \end{align}
    \label{eq:N-1 balancing}
\end{subequations}
$\pmb{\hat{g_c}}$ and $\pmb{\hat{d_c}}$ represent the adjusted values of the
generation and load after contingency. (\ref{subeq:balancing load}) sets the
load to 0 when the substation is de-energized and to its initial value otherwise.
Similarly, (\ref{subeq:balancing generation}) applies to generation, including the generation scaling factor $\sigma_{c}$ that ensures that the system is balanced \eqref{subeq:balancing overall}.

\subsection{Flows}
\label{ss:dcpf} With the base case holding the index $c=0$ and having $\pmb{\hat{d}}
_{0}=\pmb{d}$ and $\pmb{\hat{g}}_{0}=\pmb{g}$, the flows in the grid are
calculated for all cases using the classical DC power flow formulation (\ref{eq:dcpf}).

$\forall c \in \mathcal{C}$:
\begin{subequations}
    \begin{align}
        \pmb{f}_{c}\odot \pmb{w}_{c}                                              & = \pmb{0}\label{subeq:dcpf openings}                                  \\
        f_{c,e}= b_{e}\left(\phi_{c,\text{dst}(e)}- \phi_{c,\text{org}(e)}\right) & \odot (1-w_{c,e})                                                      & \forall e \in \mathcal{E}\label{subeq:dcpf flows} \\
        \pmb{A}\pmb{f}_{c}                                                        & = \pmb{\hat{g}}_{c}- \pmb{\hat{d}}_{c}\label{subeq:dcpf conservation} \\
        \left|\pmb{f}_{c}\right|                                                  & \preceq \overline{\pmb{f}}\label{subeq:dcpf limits}
    \end{align}
    \label{eq:dcpf}
\end{subequations}

When a branch is open, its flow shall be null (\ref{subeq:dcpf
openings}). The flows in the branch $e$ result from the phase angle difference between
its buses (\ref{subeq:dcpf flows}) when the branch is closed. (\ref{subeq:dcpf
conservation}) expresses the power balance in the buses (and implies \eqref{subeq:balancing overall}). Finally, (\ref{subeq:dcpf
limits}) limits the flows in the branches to their respective operational limits.

\subsection{Cost function}
\label{ss:cost function} The cost function to be minimized (\ref{eq:cost
function}) consists of the risk due to de-energized nodes. We define the risk of a single contingency $c$ as its probability $\mathfrak{p}_{c}$ of occurrence during the relevant operating window, multiplied
by the volume of the consequent load loss $\langle 1,\pmb{d}- \hat{\pmb{d}}_{c}\rangle$. Other definitions of the risk, for example, also involving duration or costs for the loss of generation could be considered.

\begin{equation}
    \min_{\pmb{v}}\sum_{c \in \mathcal{C}^\star} \mathfrak{p}_{c} \langle 1, \pmb
    {d}- \hat{\pmb{d}}_{c} \rangle \label{eq:cost function}
\end{equation}
The risk-based OTS problem minimizes the sum of contingency risks for the chosen switch configuration $\pmb{v}$.


\subsection{Big-M formulation}
\label{ss:bigM} Some equations are not pure MILP operations. This subsection gathers
a reformulation of them for an MILP solver using the big-M technique, where $M$ is
a sufficiently large scalar.

The equations involving Hadamard products of variables in the formulation (\ref{subeq:connecteness
branch opening}), (\ref{subeq:N-1 connecteness branch opening}), (\ref{subeq:balancing generation}) and (\ref{subeq:dcpf openings}), generalize
under the form $\pmb{a}= k \pmb{b}\odot \pmb{u}$, with $k$ a scalar and
$\pmb{u}$ an indicator vector and $\pmb{a}$ and $\pmb{b}$ vectors of reals. It translates
as follows:
\begin{subequations}
    \begin{align}
        \pmb{a}            & \preceq M\pmb{u}            \\
        -\pmb{a}           & \preceq M\pmb{u}            \\
        \pmb{a}- k\pmb{b}  & \preceq M(\pmb{1}- \pmb{u}) \\
        -\pmb{a}+ k\pmb{b} & \preceq M(\pmb{1}- \pmb{u})
    \end{align}
\end{subequations}

The absolute value in (\ref{subeq:dcpf limits}) translates in
\begin{subequations}
    \begin{align}
        \pmb{f}_{c} & \preceq \overline{\pmb{f}}  \\
        \pmb{f}_{c} & \preceq -\overline{\pmb{f}}
    \end{align}
\end{subequations}

The last equation to be adapted is the logical expression (\ref{subeq:N-1
connectedness pi disjunction}). To do so, the following translation of the
logical operators on binary variables will be necessary:

\begin{subequations}
    \begin{align}
        x = a \vee b   & \Leftrightarrow \begin{cases}x \ge a \\ x \ge b \\ x \le a+b\end{cases}   \\
        x = a \wedge b & \Leftrightarrow \begin{cases}x \le a \\ x \le b \\ x \ge a+b-1\end{cases}
    \end{align}\label{eq:logical operators to MILP}
\end{subequations}

First, we introduce an intermediate binary variable $\psi_{c,i,e}$ which is
defined
$\forall c \in \mathcal{C}, \forall i \in \mathcal{V}, \forall e \in \text{inc}(i
)$

\begin{equation}
    \psi_{c,i,e}= \pi_{c,\text{opp}(i,e)}\wedge (1-w_{c,e})
\end{equation}

which translates using (\ref{eq:logical operators to MILP}), into
\begin{subequations}
    \allowdisplaybreaks
    \begin{align}
        \psi_{c,i,e} & \le \pi_{c,\text{opp}(i,e)}            \\
        \psi_{c,i,e} & \le 1 - w_{c,e}\label{eq:BigM Mix-off} \\
        \psi_{c,i,e} & \ge \pi_{c,\text{opp}(i,e)}- w_{c,e}
    \end{align}
\end{subequations}

Now (\ref{subeq:N-1 connectedness pi disjunction}) becomes
\begin{equation}
    \pi_{c,i}= \bigvee_{e \in \text{inc}(i)}\psi_{c,i,e}\quad \forall (c,i) \in \mathcal{C}
    \times\mathcal{V}
\end{equation}

which translates using (\ref{eq:logical operators to MILP}) into

$\forall (c,i) \in \mathcal{C}\times\mathcal{V}$:
\begin{subequations}
    \allowdisplaybreaks
    \begin{align}
        \pi_{c,i} & \le \sum_{e \in \text{inc}(i) }\psi_{c,i,e} \\
        \pi_{c,i} & \ge \psi_{c,i,e}                           & \forall e \in \text{inc}(i)
    \end{align}
\end{subequations}